\documentclass[11pt]{article}

\usepackage{times}
\usepackage{epsf}
\usepackage{epsfig}
\usepackage{amsmath, alltt, amssymb, xspace}
\usepackage{wrapfig}
\usepackage{fancyhdr}
\usepackage{url}
\usepackage{verbatim}
\usepackage{fancyvrb}

\usepackage{subfigure}
\usepackage{cite}
%\usepackage{cases}
%\usepackage{ltexpprt}
%\usepackage{verbatim}

%\topmargin      -0.70in  % distance to headers
%\headheight     0.2in   % height of header box
%\headsep        0.4in   % distance to top line
%\footskip       0.3in   % distance from bottom line

% Horizontal alignment
\topmargin      -0.50in  % distance to headers
\oddsidemargin  0.0in
\evensidemargin 0.0in
\textwidth      6.5in
\textheight     8.9in 


%\centerfigcaptionstrue

%\def\baselinestretch{0.95}


\newcommand\discuss[1]{\{\textbf{Discuss:} \textit{#1}\}}
%\newcommand\todo[1]{\vspace{0.1in}\{\textbf{Todo:} \textit{#1}\}\vspace{0.1in}}
\newtheorem{problem}{Problem}[section]
%\newtheorem{theorem}{Theorem}
%\newtheorem{fact}{Fact}
\newtheorem{define}{Definition}[section]
%\newtheorem{analysis}{Analysis}
\newcommand\vspacenoindent{\vspace{0.1in} \noindent}

%\newenvironment{proof}{\noindent {\bf Proof}.}{\hspace*{\fill}~\mbox{\rule[0pt]{1.3ex}{1.3ex}}}
%\newcommand\todo[1]{\vspace{0.1in}\{\textbf{Todo:} \textit{#1}\}\vspace{0.1in}}

%\newcommand\reducespace{\vspace{-0.1in}}
% reduce the space between lines
%\def\baselinestretch{0.95}

\newcommand{\fixmefn}[1]{ \footnote{\sf\ \ \fbox{FIXME} #1} }
\newcommand{\todo}[1]{
\vspace{0.1in}
\fbox{\parbox{6in}{TODO: #1}}
\vspace{0.1in}
}

\newcommand{\mybox}[1]{
\vspace{0.2in}
\noindent
\fbox{\parbox{6.5in}{#1}}
\vspace{0.1in}
}


\newcounter{question}
\setcounter{question}{1}

\newcommand{\myquestion} {{\noindent \bf Question \arabic{question}:} \addtocounter{question}{1} \,}



\newcommand{\copyrightnotice}[1]{
\vspace{0.1in}
\fbox{\parbox{6in}{\small Copyright \copyright\ 2006 - 2009\ \ Wenliang Du, Syracuse University.\\ 
      The development of this document is funded by 
      the National Science Foundation's Course, Curriculum, and Laboratory 
      Improvement (CCLI) program under Award No. 0618680 and 0231122. 
      Permission is granted to copy, distribute and/or modify this document
      under the terms of the GNU Free Documentation License, Version 1.2
      or any later version published by the Free Software Foundation.
      A copy of the license can be found at http://www.gnu.org/licenses/fdl.html.}}
\vspace{0.1in}
}

\newcommand{\nocopyrightnotice}[1]{
\vspace{0.1in}
\fbox{\parbox{6in}{\small  
      The development of this document is funded by 
      the National Science Foundation's Course, Curriculum, and Laboratory 
      Improvement (CCLI) program under Award No. 0618680 and 0231122. 
      Permission is granted to copy, distribute and/or modify this document.
      }}
\vspace{0.1in}
}

\newcommand{\idea}[1]{
\vspace{0.1in}
{\sf IDEA:\ \ \fbox{\parbox{5in}{#1}}}
\vspace{0.1in}
}


\newcommand{\minix}{{\tt Minix}\xspace}
\newcommand{\unix}{{\tt Unix}\xspace}
\newcommand{\linux}{{\tt Linux}\xspace}
\newcommand{\ubuntu}{{\tt Ubuntu}\xspace}
\newcommand{\selinux}{{\tt SELinux}\xspace}
\newcommand{\freebsd}{{\tt FreeBSD}\xspace}
\newcommand{\solaris}{{\tt Solaris}\xspace}
\newcommand{\windowsnt}{{\tt Windows NT}\xspace}
\newcommand{\setuid}{{\tt Set-UID}\xspace}
%\newcommand{\smx}{{\tt Smx}\xspace}
\newcommand{\smx}{{\tt Minix}\xspace}
\newcommand{\relay}{{\tt relay}\xspace}
\newcommand{\isys}{{\tt iSYS}\xspace}
\newcommand{\ilan}{{\tt iLAN}\xspace}
\newcommand{\iSYS}{{\tt iSYS}\xspace}
\newcommand{\iLAN}{{\tt iLAN}\xspace}
\newcommand{\iLANs}{{\tt iLAN}s\xspace}
\newcommand{\bochs}{{\tt Bochs}\xspace}

\newcommand\FF{{\mathcal{F}}}

\newcommand{\argmax}[1]{
\begin{minipage}[t]{1.25cm}\parskip-1ex\begin{center}
argmax
#1
\end{center}\end{minipage}
\;
}

\newcommand{\bm}{\boldmath}
\newcommand  {\bx}    {\mbox{\boldmath $x$}}
\newcommand  {\by}    {\mbox{\boldmath $y$}}
\newcommand  {\br}    {\mbox{\boldmath $r$}}


%\pagestyle{fancyplain}
%\lhead[\thepage]{\thesection}      % Note the different brackets!
%\rhead[\thesection]{SEED Laboratories}
%\lfoot[\fancyplain{}{}]{Syracuse University} 
%\cfoot[\fancyplain{}{}]{\thepage} 

\newcommand{\tstamp}{\today}   
%\lhead[\fancyplain{}{\thepage}]         {\fancyplain{}{\rightmark}}
%\chead[\fancyplain{}{}]                 {\fancyplain{}{}}
%\rhead[\fancyplain{}{\rightmark}]       {\fancyplain{}{\thepage}}
%\lfoot[\fancyplain{}{}]                 {\fancyplain{\tstamp}{\tstamp}}
%\cfoot[\fancyplain{\thepage}{}]         {\fancyplain{\thepage}{}}
%\rfoot[\fancyplain{\tstamp} {\tstamp}]  {\fancyplain{}{}}

\pagestyle{fancy}
%\lhead{\bfseries Computer Security Course Project}
\lhead{\bfseries Laboratory for Computer Security Education}
\chead{}
\rhead{\small \thepage}
\lfoot{}
\cfoot{}
\rfoot{}


\def \code#1 {\fbox{\scriptsize{\texttt{#1}}}}

\begin{document}

\begin{center}
{\LARGE Race Condition Vulnerability Lab}
\end{center}

\copyrightnotice

\section{Lab Overview}

The learning objective of this lab is for students to gain the first-hand
experience on the race-condition vulnerability by putting what they have learned
about the vulnerability from class into actions.
A race condition occurs when multiple processes access and manipulate the same
data concurrently, and the outcome of the execution depends on the particular
order in which the access takes place. If a privileged program has a race-condition 
vulnerability, attackers can run a parallel process to ``race'' against
the privileged program, with an intention to change the behaviors 
of the program.

In this lab, students will be given a program with a race-condition
vulnerability; their task is to develop a scheme to exploit 
the vulnerability and gain the root privilege.  In addition to the
attacks, students will be guided to walk through several protection
schemes that can be used to counter the race-condition attacks.
Students need to evaluate
whether the schemes work or not and explain why.

\section{Lab Tasks}


\subsection{A Vulnerable Program}


The following program is a seemingly harmless program. It contains a race-condition
vulnerability. 

\begin{verbatim}
    /*  vulp.c  */

    #include <stdio.h>
    #include<unistd.h>

    #define DELAY 10000

    int main()
    {
       char * fn = "/tmp/XYZ";
       char buffer[60];
       FILE *fp;
       long int  i;

       /* get user input */
       scanf("%50s", buffer );

       if(!access(fn, W_OK)){
           /* simulating delay */
            for (i=0; i < DELAY; i++){
               int a = i^2; 
            }

            fp = fopen(fn, "a+");
            fwrite("\n", sizeof(char), 1, fp);
            fwrite(buffer, sizeof(char), strlen(buffer), fp);
            fclose(fp);
       }
       else printf("No permission \n");
    }
\end{verbatim}

This is part of a \setuid program (owned by {\tt root}); 
it appends a string of user input to
the end of a temporary file {\tt /tmp/XYZ}. Since the code runs
with the root privilege, it carefully 
checks whether the real user actually has the access permission to the file
{\tt /tmp/XYZ}; that is the purpose of the {\tt access()} call.
Once the program has made sure that the real user indeed has 
the right, the program opens the file and writes the user
input into the file. 

It appears that the program does not have any problem at the first look. 
However, there is a race condition vulnerability in this program: due to the 
window (the simulated delay) between the check ({\tt access}) and 
the use ({\tt fopen}), there is a possibility that the file used by
{\tt access} is different from the file used by {\tt fopen}, even
though they have the same file name {\tt /tmp/XYZ}.  If a malicious 
attacker can somehow make {\tt /tmp/XYZ} a symbolic link pointing to
{\tt /etc/shadow}, the attacker can cause the user input
to be appended to {\tt /etc/shadow} (note that the program runs with
the root privilege, and can therefore overwrite any file).


\subsection{Task 1: Exploit the Race Condition Vulnerabilities}

You need to exploit the race condition vulnerability in the above 
\setuid program. More specifically, you need to achieve the followings:

\begin{enumerate}
\item Overwrite any file that belongs to {\tt root}.

\item Gain root privileges; namely, you should be able to do anything 
      that root can do.  
\end{enumerate}

%If you observe closely, there is a delay between the time when the access
%permission on the file is checked and the time when it is used. This time delay
%is used by the attacker to his advantage. This vulnerability is also known as
%TOCTOU (time of check time of use). We will exploit the above program to gain
%root access on the system. Since the above program allows us to write a string
%into a file, we need to make changes into some files such that we get root
%access on the system. 


\subsection{Task 2: Protection Mechanism A: Repeating}

Getting rid of race conditions is not easy, because the check-and-use pattern 
is often necessary in programs. Instead of removing race conditions,
we can actually add more race conditions, such that to compromise the 
security of the program, attackers need to win all these race conditions.
If these race conditions are designed properly, we can exponentially 
reduce the winning probability for attackers. The basic idea is to
repeat {\tt access()} and {\tt open()} for several times; at each time, 
we open the file, and at the end, we check whether the same file is opened
by checking their {\tt i-nodes} (they should be the same).

Please use this strategy to modify the vulnerable program, and repeat 
your attack. Report how difficult it is to succeed, if you can still
succeed.

\subsection{Task 3: Protection Mechanism B: Principle of Least Privilege}

The fundamental problem of the vulnerable program in this lab is 
the violation of the {\em Principle of Least Privilege}. 
The programmer does understand that the user who runs the program 
might be too powerful, so he/she introduced {\tt access()} to limit the user's 
power. However, this is not the proper approach. A better
approach is to apply the {\em Principle of Least Privilege}; 
namely, if users do not need certain privilege, the privilege
needs to be disabled.

We can use {\tt seteuid{}} system call to temporarily disable
the root privilege, and later enable it if necessary. Please use 
this approach to fix the vulnerability in the program, and then
repeat your attack. Will you be able to succeed? Please report your
observations and explanation.



\section{Guidelines} 

\subsection{Two Potential Targets}

There are possibly many ways to exploit the race condition vulnerability
in {\tt vulp.c}. One way is to use the vulnerability to append some 
information to both {\tt /etc/passwd} and {\tt /etc/shadow}. These two
files are used by \unix operating systems to authenticate users. If attackers
can add information to these two files, they essentially have the power
to create new users, including super-users (by letting uid to be zero).

The {\tt /etc/passwd} file is the authentication database for a Unix machine. 
It contains basic user attributes. 
This is an ASCII file that contains an entry for each
user. Each entry defines the basic attributes applied to a user. When you use
the {\tt mkuser} command to add a user to your system, the command updates the
{\tt /etc/passwd} file. 

The file {\tt /etc/passwd} has to be world readable, because many application
programs need to access user attributes, such as user-names, home directories, etc.
Saving an encrypted password in that file would mean that
anyone with access to the machine could use password cracking programs (such as
{\tt crack}) to break into the accounts of others. To fix this problem, the shadow
password system was created. The {\tt /etc/passwd} file in the shadow system is
world-readable but does not contain the encrypted passwords. Another file,
{\tt /etc/shadow}, which is readable only by root contains the passwords.

To find out what strings to add to these two files, run {\tt mkuser}, and see
what are added to these files. For example, the followings are what have 
been added to these files after creating a new user called {\tt smith}: 

\begin{verbatim}

/etc/passwd: 
-------------
  smith:x:1000:1000:Joe Smith,,,:/home/smith:/bin/bash

/etc/shadow: 
-------------
  smith:*1*Srdssdsdi*M4sdabPasdsdsdasdsdasdY/:13450:0:99999:7:::
\end{verbatim}

The third column in the file {\tt /etc/passwd} denotes the UID of the user. 
Because {\tt smith} account is a regular user account, its value 1000 is nothing 
special. If we change this entry to 0, {\tt smith} now becomes {\tt root}. 


\subsection{Creating symbolic links} 

You can manually create symbolic links using {\tt "ln -s"}. You can also call
C function {\tt symlink} to create symbolic links in your program. 
Since \linux does not allow one to create a link if the link already exists, 
we need to delete the old link first. 
The following C code snippet shows how to remove a link and then make 
{\tt /tmp/XYZ} point to {\tt /etc/passwd}:

\begin{verbatim}
  unlink("/tmp/XYZ");
  symlink("/etc/passwd","/tmp/XYZ");
\end{verbatim}


\subsection{Improving success rate}

The most critical step (i.e., pointing the link to our 
target file) of a race-condition attack must occur within
the window between check and use; namely between the {\tt access} and the {\tt fopen} 
calls in {\tt vulp.c}. Since we cannot modify the vulnerable program,
the only thing that we can do is to run our attacking program in parallel
with the target program, hoping that the change of the link does occur
within that critical window.
Unfortunately, we cannot achieve the perfect timing. Therefore, 
the success of attack is probabilistic.
The probability of successful attack might be quite low if the window
are small. You need to think about how to increase the 
probability (Hints: you can run the vulnerable program for many
times; you only need to achieve success once among all these trials).


Since you need to run the attacks and the vulnerable program for many
times, you need to write a program to automate the attack process. 
To avoid manually typing an input to {\tt vulp}, you can use 
redirection. Namely, you type your input in a file, and then
redirect this file when you run {\tt vulp}. For example,
you can use the following: {\tt vulp < FILE}.

In the program {\tt vulp.c}, we intentionally added a DELAY parameter
in the program. This is intended to make your attack easier. Once you have 
succeeded in your attacks, gradually reduce the value for DELAY. When DELAY becomes
zero, how much longer does it take you to succeed? 


\subsection{Knowing whether the attack is successful}

Since the user does not have the read permission for accessing {\tt /etc/shadow},
there is no way of knowing if it was modified. The only way that is possible is
to see its time stamps. Also it would be better if we stop the attack once the
entries are added to the respective files. The following shell script checks if
the time stamps of {\tt /etc/shadow} has been changed. 
It prints a message once the change is noticed. 

\begin{verbatim}
        #!/bin/sh

        old=`ls -l /etc/shadow`
        new=`ls -l /etc/shadow`
        while [ "$old" = "$new" ]
        do
            new=`ls -l /etc/shadow`
        done
        echo "STOP... The shadow file has been changed"
\end{verbatim}

\subsection{Troubleshooting}

While testing the program, due to untimely killing of the attack program,
{\tt /tmp/XYZ} may get into an unstable state. When this happens the OS automatically
makes it a normal file with root as its owner. If this happens, the file has to
be deleted and the attack has to be restarted.

\subsection{Warning}

In the past, some students accidentally emptied the {\tt /etc/shadow} file 
during the attacks (we still do not know what has caused that). If you lose
the shadow file, you will not be able to login again. To avoid this 
trouble, please make a copy of the original shadow file. 


\end{document}
